\documentclass{article}
\usepackage[utf8]{inputenc}
\usepackage[russian]{babel}
\usepackage{hyperref}
\usepackage{underscore}
\usepackage{setspace}
\usepackage{indentfirst} 

\usepackage[left=1.4cm,right=1.4cm,
    top=2.3cm,bottom=2.3cm,bindingoffset=0cm]{geometry}

\singlespacing

\begin{document}

\section{Славяне Восточной Европы и их соседи до IX в.}
Ответ

\section{Образование Древнерусского государства.}
Ответ

\section{Древнерусское государство в X -- начале XI века. Крещение Руси.}
Ответ

\section{Древнерусское государство в XI -- начале XII века. Ярослав Мудрый. Владимир Мономах.}
Ответ

\section{Политическая раздробленность на Руси в XII -- начале XIII века. Новгородская земля.}
Ответ

\section{Политическая раздробленность на Руси в XII -- начале XIII века. Владимиро-Суздальское княжество.}
Ответ

\section{Культура Руси домонгольского периода.}
Ответ

\section{Монгольское нашествие и его последствия для Руси.}
Ответ

\section{Борьба Руси со шведской и немецкой агрессией в XIII в. Александр Невский.}
Ответ

\section{Предпосылки образования единого Русского государства. Москва и Тверь в конце XIII -- первой половине XIV в.}
Ответ

\section{Русские земли и княжества середины XIV -- середины XV вв. Дмитрий Донской.}
Ответ

\section{Золотая Орда и Великое княжество Литовское в XIII -- XV вв.}
Ответ

\section{Завершение объединения русских земель при Иване III и Василии III.}
Ответ

\section{Русская культура XIII -- XV вв.}
Ответ

\section{Русское государство в XVI в.: территория и население, политическое и социальное устройство.}
Ответ

\section{Политическая борьба в 30 -- 40-е гг. XVI в. Реформы ``Избранной рады''}
Ответ

\section{Опричнина Ивана Грозного.}
Ответ

\section{Внешняя политика Русского государства в XVI в.}
Ответ

\section{Русская культура XVI в.}
Ответ

\section{Русское государство на рубеже XVI -- XVII вв. Причины Смуты.}
Ответ

\section{Смута в начале XVII в. Лжедмитрий I. Лжедмитрий II. Василий Шуйский. Иностранная интервенция, первое и второе ополчения.}
Ответ

\section{Российское государство в правление Михаила Федоровича Романова.}
Ответ

\section{Россия в середине и второй половине XVII в.: политический строй и социально-экономическое развитие. Царь Алексей Михайлович.}
Ответ

\section{Церковная реформа патриарха Никона. Народные движения середины и второй половины XVII в.}
Ответ

\section{Внешняя политика России в середине и второй половине XVII в.}
Ответ

\section{Русская культура XVII в.}
Ответ

\section{Россия в конце XVII в. Царь Федор Алексеевич. Царевна Софья. Начало правления Петра I.}
Ответ

\section{Внешняя политика России при Петре I.}
Ответ

\section{Социально-экономическая политика Петра I. Народные движения.}
Ответ

\section{Реформы государственного аппарата в первой четверти XVIII в.}
Ответ

\section{Культурные переворот петровского времени.}
Ответ

\section{Россия в эпоху дворцовых переворотов (1725 -- 1741 гг.)}
Ответ

\end{document}