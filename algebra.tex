\documentclass{article}
\usepackage[utf8]{inputenc}
\usepackage[russian]{babel}
\usepackage{hyperref}
\usepackage{underscore}
\usepackage{setspace}
\usepackage{indentfirst} 
\usepackage{mathtools}
\singlespacing

\begin{document}

\section*{Ранг матрицы}
Пусть дана матрица
\[
A =
\left(
\begin{matrix}
    a_{11} & a_{12} & \hdots & a_{1n} \\
    a_{21} & a_{22} & \hdots & a_{2n} \\
    \vdots & \vdots & \ddots & \vdots \\
    a_{s1} & a_{s2} & \hdots & a_{sn} \\
\end{matrix}  
\right)  
\]
содержащая $s$ строк и $n$ столбцов.

Ранг системы столбцов, т.е. максимальное число линейно независимых столбцов
матрицы $A$ (точнее, число столбцов, входящих в любую максимальную линейно
независимую подсистему системы столбцов), называется \textit{рангом} этой матрицы.

Ранг системы строк матрицы равен рангу системы её столбцов, т.е. равен рангу этой матрицы.

Выбираем в матрице $A$ произвольные $k$ строк и $k$ столбцов, $k \le \min(s, n)$.
Элементы, стоящие на пересечении этих строк и столбцов, составляют квадратную
матрицу $k$-го порядка, определитель которой называется \textit{минором $k$-го порядка}
матрицы $A$. Дальше нас будут интересовать порядки тех миноров матрицы $A$, которые
отличны от нуля, а именно \textbf{наивысший среди этих порядков}. При его разыскании
полезно учитывать следующее замечание: \textit{если все миноры $k$-го порядка матрицы
$A$ равны нулю, то равны нулю и все миноры более высоких порядков}.

\textbf{Наивысший порядок отличных от нуля миноров матрицы $A$ равен рангу этой матрицы}.

При вычислении ранга матрицы следует переходить от миноров меньших порядков к минорам
больших порядков. Если уже найден минор $k$-го порялка $D$, отличный от нуля, то
требуют вычисления лишь миноры $(k + 1)$-го порядка, окаймляющие минор $D$:
если все они равны нулю, то ранг матрицы равен $k$.

Максимальное число линейно независимых строк всяков матрицы равно максимальному числу
её линейно независимых столбцов, т.е. равно рангу этой матрицы.

Определитель $n$-го порядка тогда и только торда равен нулю, если между его строками
существует линейная зависимость.

\textit{Элементарными преобразованиями} матрицы $A$ называются следующие 
преобразования этой матрицы:
\begin{enumerate}
    \item перемена мест (транспозиция) двух строк или двух столбцов;
    \item умножение строки (или столбца) на произвольное отличное от нуля число;
    \item прибавление к одной строке (или столбцу) другой строки (столбца),
    умноженной на некоторое число.
\end{enumerate}

\textit{Элементарные преобразования не меняют ранга матрицы.}

Говорят, что матрица, содержащая $s$ строк и $n$ столбцов, имеет
\textit{диагональную форму}, если все её элементы равны нулю, кроме
элементов $a_{11},a_{22},\dots,a_{rr}$ (где $0 \le r \le \min (s, n)$).
равных единице. Ранг этой матрицы равен $r$.

Всякую матрицу можно элементарными преобразованиями привести к диагональной форме.

Таким обазом, для нахождения ранга матрицы нужно элементарными преобразованиями
привести эту матрицу к диагональной форме и подсчитать число единиц, стоящих
в последней на главной диагонали.


\end{document}