\section{Машинно-зависимые языки программирования}

Выделяют две основные сферы применения машинно зависимых языков:
\begin{enumerate}
    \item создание системных программ, входящих в состав драйверов нестандартных машинных устройств
    \item решение специфических задач для информационных и управляющих систем, например систем реального времени.
\end{enumerate}

При классификации программных средств, традиционное их делят на
\begin{itemize}
    \item Прикладные (пользовательские), ориентированные на конечного пользователя
    \item Системные, поддерживающие работу вычислительной системы в автоматическом режиме
\end{itemize}

Прикладные программы включают в себя пакеты прикладных программ, библиотеки объектных модулей
которые используются для решения задач, в конкретной области применения техники.

К классу системных программ, относятся программы, которые обеспечивают автоматический режим работы вычислительной системы
а также разработку, модификацию различных программ, как прикладных, так и системных.

Управляющие системные программы, обеспечивающие корректное выполнение всех процессов
при решении задач и функционировании системы постоянно сохраняются в оперативной памяти,
составляют ядро операционной системы и называются резидентными.

Управляющие программы, которые подргужаются в оперативную память по мере необходимости, перед их выполнением
называются транзитными. 

Обрабатывающие системные программы выполняются, как специальные проблемные или приложения операционных систем
используемых при разработке новых и модификации уже разработанных программ.

разработка системных программ, требует использование специальных системных языков программирования, это могут быть машинно-зависимые языки "--- <<ассемблеры>> или такие языки как С или PLM, а также языки
высокого уровня, которые имеют развитые средства работы с внутренней адресной информацией.

Кроме того, многие языки высокого уровня имеют возможность использования ассемблерных 
вставок, поэтому говорят, что языки ассемблера и языки низкого уровня сохраняют свое значение, в областях, 
где требуется высокая скорость выполнения кода,
и будут использоваться до тех пор, пока идет процесс совершенстования архитектур процессоров.

Поэтому говорят, что на ассемблере пишут все что требует максимальной скорости вычислений: основные компоненты игр, ядра ОС, многозадачных систем, в том числе реального времени,
все, что непосредственно действует с внешними устройствами, драйвера для нестандартных устройств, 
программы перевода в защищенный режим, чтобы полностью использовать возможности ОС.

Одна из сфер применения, это обработка больших объемов информации.
недостатки - трудно выучить, говорят, трудно читаемые, говорят, но вы держитесь, не переносится на другие процессоры, трудно писать, нет стандартных модулей
Архитектуры ПК
архитектура ПК включает в себя - структурную организацию, т.е организацию аппартных средств, наличие блоков у-св, а так-же функциональную организацию
позволяющую огранизовать программное управление этой ВС. С точки зрения программиста, архитектура ПК это совокупность программно доступных ср-в, 
с этой точки зрения архитектуру пк можно представить след образом
ПК арх строится на магистрально модульном принципе, который заключается в том что к системной шине в центральной магистрали стандартным образом, с помощью стандартного интерфейса 
подключаются все устройства
системная шина включает в себя шину данных, адресную шину, шину управления
шины - это набор линий связи, по которым информация передается от одного из участников к одному или нескольким приемникам
внешние ус-ва работают значительно медленне ЦП, поэтому в архитектуру ПК подключен, не отраженный на схеме канал прямого доступа к памяти, 
а для связи с внешними ус-вами используется интерфейсные блоки, это по-сути устройства управления внешними устройствами
который содержит в себе собственные шины данных, шины управления, адресную шину
пару слов о шинах
адресная шина однонаправленная, адреса ?? 
шина данных, двунаправленная данные и от процессора и к процессору
а шина управления содержит как однонаправленные, как и двунаправленные
(канал прямого доступа к памяти и интерфейсные блоки, позволяют обеспечить параллельную работу процессора и внешних ус-в
но для синхронизации действий всех ус-в устр исп система прерываний в схему архитектуры пк включен контроллер прерываний,???)
он может быть один, или каскадный контроллер, в зависимости от количества подключенных внешних ус-в
и организация работы ВС может осуществляться следующим образом - 
когда какому либо устройству требуется работа процессора, например ввести данные или вывести
это устройство посылает процессору специальный сигнал прерывания, он проходит через контроллер прерываний, если процессор может обслужить прерывание
этот сигнал передается процессору, процессор его изучае и возвращает его с именем int a затем формируется номер прерывание, процессор прерывает выполнение текущей программы
запомнив предварительно следующую команду, которая должна была выполнятся и передает управление на выполнение программы обработки выполнения этого прерывания,
если прерывание успешно обработано, то процессор продолжает выполнение прерванной программы.
прерывания бывают различные, это могут быть внешние/внутренние прерывания, системные, программные, маскированные/немаскированные
начальный адресат программ обработки прерываний хранятся в спец таблице вектора прерываний

архитектура ix86 ( там на слайде, я отдохну )

бля

( опять слайд )

???
AX-аккумулятор, в нем хранится результат, если в один регистр не умещается, то старшая часть DX - регичстр данных
BX - базовый ?? CX - счетчик, исп ?? в командах сдвига, регистр сп и бп исп при работе со стеком
SI DI, индекс источника, индекс приемника, по умолчанию при работе со строками и для сложной адресации операнда
Счетчик команд IP
регистр определяет состояние команд в каждый текущий момент времени
сегментные регистры используются для определения адресов начала сегментов
???