\section{Комбинаторика}

\subsection{Аксиоматика. Размещения, сочетания, перестановки.}

\subsubsection{Правила сложения и умножения}

\begin{theorem}[Правило сложения]
    Пусть имеется возможность выбрать элемент $a$ из $n$"=элементного
    множества $A$,
    а элемент $b$ из $m$"=элементного множества $B$. Тогда выбрать элемент 
    из $A$ или $B$ можно $n + m$ способами.
\end{theorem}

\begin{example}
    Выбрать книгу или видеоигру из 15 книг и 3 видеоигр можно 
    $15 + 3 = 18$ способами.
\end{example}

\begin{theorem}[Правило умножения]
    Пусть имеется возможность выбрать элемент $a$ из $n$"=элементного
    множества $A$ и элемент $b$  из $m$"=элементного множества $B$. 
    Тогда последовательно выбрать $a$ из $A$, а потом выбрать 
    $b$ из $B$ можно $n \times m$ способами.
\end{theorem}

\begin{example}
    Выбрать одну книгу и одну видеоигру из 15 книг и 3 видеоигр можно
    $15 \times 3 = 45$ способами.
\end{example}

\subsubsection{Сочетания, размещения, перестановки}
Пусть имеется набор $A = \{a_1, a_2, \dots, a_n\}$. Попробуем понять, 
каким образом мы можем извлекать $k$ элементов из этого набора.

Во"=первых, мы можем подобно правилу умножения <<последовательно>> извлекать 
элементы из множества. Сначала некоторый первый элемент, затем второй и так далее.
В рамках такого случая также есть два возможных варианта поведения.
\begin{itemize}
    \item мы могли бы <<извлечь>> объект, запомнить его и <<вернуть>> в исходное множество. Очевидно, что в этом случае $k \in N$
    \item мы могли бы <<извлечь>> объект и не <<возвращать>> его в исходное множество. Тем самым,
    производя следующий выбор из оставшихся элементов множества. Очевидно, что в таком случае
    $k \leq n$.
\end{itemize}

Оба этих варианта описывают общий случай, называемый размещениями. Итак.

\begin{definition}
    Размещением из $n$ по $k$ элементов назовем упорядоченный набор
    из $k$ различных элементов из некоторого множества различных $n$ элементов.
\end{definition}

В первом случае при этом мы осуществляем размещение без повторений
и обозначаем это как $A_n^k$. Во втором случае размещение с повторениями
и обозначаем это как $\overline{A_n^k}$.

Стоит отметить о частном случае размещения без повторений "--- перестановке.
В этом случае $k = n$ ($A_n^n$)

Теперь рассмотрим другой способ. Мы могли бы брать элементы не последовательно,
а некоторой <<пригоршней>>, то есть не учитывать порядок, в котором эти
элементы были взяты. Введем определение.

\begin{definition}
    Сочетанием из $n$ по $k$ назовем набор из $k$ элементов таких, что
    эти наборы отличаются друг от друга только самими элементами, но не их
    порядком.
\end{definition}

А могут ли сочетания быть с повторениями и без повторений? Да. Что есть 
сочетание без повторений ясно, но что в случае с повторениями? В таком случае
мы предпологаем, что каждый объект множества присутствует в выборке в нескольких
экзеплярах.

Сочетание без повторений обозначим как $C_n^k$, сочетание с повторениями 
"--- $\overline{C_n^k}$.

\begin{example}
    В качестве примера возьмем замечательное слово <<ЛЯГУШКА>>. Все буквы
    в нем встречаются один раз. Это пример \textit{последовательного выбора без повторений (размещения без повторений)}.
    
    Мы могли бы очень сильно впечатлиться и сказать <<ЛЯГУУУУУУУУУШКА>>.
    В таком случае мы будем иметь дело с \textit{последовательным выбором с повторениями (размещением с повторениями)}.
    
    Если же мы рассмотрим наше слово в виде $\{\text{Л, Я, Г, У, Ш, К, А}\}$ (в виде набора букв)
    и для нас не будет иметь значения, в каком порядке эти буквы были выбраны, то мы будем
    иметь дело с \textit{сочетаниями без повторений}. 
    Понятно, что в таком случае между
    наборами $\{\text{Л, Я, Г, У, Ш, К, А}\}$ и $\{\text{Г, У, Л, Я, Ш, К, А}\}$ нет никакой разницы.
    
    Аналогично мы могли бы рассмотреть и случай сочетаний с повторениями.
\end{example}

Теперь получим формулы для вычисления числа возможных способов
для каждого случая.

\begin{theorem}[О числе размещений с повторениями]
    $\overline{A_n^k} = n^k$
\end{theorem}

\begin{proof}
    Пусть имеется некоторое множество $A = \{a_1, a_2, \dots, a_n\}$
    и мы хотим извлечь из него $k$ элементов с повторениями. Этот процесс
    представляет из себя последовательный выбор одного элемента из $n$ возможных
    (т.к элемент после выбора <<воззвращается на место>>). Этот случай
    в чистом виде представляет из себя правило умножение. Получим:
    \begin{equation*}
        \displaystyle \overline{A_n^k} = \underbrace{n \times n \times n \times \dots}_{\text{k раз}} = n^k
    \end{equation*}
\end{proof}

\begin{theorem}[О числе размещений без повторений]
    $\displaystyle A_n^k = \frac{n!}{(n - k)!}$
\end{theorem}

\begin{proof}
    Пусть имеется некоторое множество $A = \{a_1, a_2, \dots, a_n\}$.
    Выберем из него первый из $k$ элементов. В отличие от размещения с повторениями
    мы не будем <<возвращать>> элемент обратно в исходное множество.
    Тогда, оставшиеся $k - 1$ элемент мы можем выбрать $n - 1$ способами.
    Продолжая эту логику далее для оставшихся $k - 1$ элементов получим, что
    \begin{equation*}
        \displaystyle A_n^k = n \times (n - 1) \times (n - 2) \times \dots \times (n - k + 1) = \frac{n!}{(n - k)!}
    \end{equation*}
\end{proof}

Отметим очевидный факт, что число перестановок $\displaystyle A_n^n = n!$.

\begin{theorem}[О числе сочетаний без повторений]
    $\displaystyle C_n^k = \frac{n!}{k!(n - k)!}$
\end{theorem}

\begin{proof}
    Рассмотрим некоторое множество $A = \{a_1, a_2, \dots, a_n\}$. Зафиксируем в нем конкретный набор $C_n^k = \{a_1, a_2, \dots, a_k\}$.
    Отметим, что $k$"=перестановкой этого сочетания мы получим 
    число размещений этого набора. То есть, $C_n^k \times k! = A_n^k$.
    Тогда
    \begin{equation}
     C_n^k = \frac{n!}{k!(n - k)!}
    \end{equation} 
\end{proof}

\begin{theorem}[О числе сочетаний с повторениями]
    $\displaystyle \overline{C_n^k} = C_{n + k - 1}^k$
\end{theorem}

\begin{proof}
    Воспользуемся известным методом доказательства с помощью 
    <<шариков с перегородками>>. Для некоторого множества $A = \{a_1, a_2, \dots, a_n\}$
    зафиксирует $k$"=сочетание с повторениями. Каждый элемент множества $a$ может 
    встретиться в этом сочетании от $0$ до $k$ раз. 
    Запишем цифру $1$ столько раз, сколько раз $a_1$ встречается
    в сочетании. После всех таких единиц поставим $0$. 
    
    Затем запишем цифру
    $1$ столько раз, сколько раз встречается $a_2$, после чего снова
    поставим $0$. Продолжим эту операцию до $a_n$, при этом не будем ставить $0$
    ни перед $a_1$, ни после $a_n$.
    
    Получим некоторую последовательность: $\underbrace{11\dots1}_{\text{число a_1}}0
    \underbrace{1\dots1}_{\text{число a_2}}0\dots\underbrace{11\dots1}_\text{число a_n}$

    Эта последовательность содержит $n - 1$ нулей и $k$ единиц. Тем самым, 
    мы получили биекцию между множеством всех возможных $k$"=сочетаниями с повторениями и множеством последовательностяей
    с заданными нами условиями. 
    
    Число интересующих нас сочетаний при этом в точности
    равно числу последовательностей длины $n - k + 1$, в которых ровно $k$ единиц.
    Осталось отметить, что это число равно $C_{n - k + 1}^n$
\end{proof}






\begin{theorem}[Принцип Дирихле]
    Пусть имеется $n + 1$ <<кроликов>> и $n$ <<ящиков>>. 
    Если всех <<кроликов>> рассадить по <<ящикам>>, 
    то найдется <<ящик>> с не менее двумя <<кроликами>>.
\end{theorem}

Принцип Дирихле во многом является основопологающим при доказательстве многих
нетривиальных фактов из мира математики. Основным навыком при этом является
умение придумать, что в конкретной задаче является <<кроликами>>, а что
<<ящиками>>.
