\section{Численные множества}

\subsection{Точные грани численного множества}

\begin{itemize}
    \item $[a;b]$ -- отрезок $= \{x | a \le x \le b\}$
    \item $(a;b)$ -- интервал $= \{x | a < x < b\}$
    \item $(a;b]$ -- полуинтервал $= \{x | a < x \le b\}$
    \item $[a;b)$ -- полуинтервал $= \{x | a \le x < b\}$
\end{itemize}

$\mathbb{N} \subset \mathbb{Z} \subset \mathbb{Q} \subset \mathbb{R}$

\subsection{Ограниченность}

Числовое множество $A$ называют ограниченным сверху, если 
$\exists M \in \mathbb{R}: \forall x \in A$ выполняется $x \le M$,
число $M$ называется \textbf{мажорантой} множества $A$.

Числовое множество $A$ называют ограниченным снизу, если 
$\exists m \in \mathbb{R}: \forall x \in A$ выполняется $x \ge m$,
число $m$ называется \textbf{минорантой} множества $A$.

Множество $A$ называется ограниченным, если оно ограничено сверху или снизу,
т.е. $\exists m, M \in \mathbb{R}: \forall x \in A$ выполняется $m \le x \le M$
(или $\exists C > 0: \forall x \in A$ выполняется $|x| \le C$).

\begin{definition}
    Наименьшая мажоранта множества $A$ называется верхней гранью $A$ и обозначается
    $\sup A$ (супремум).

    Наибольшая миноранта множества $A$ называется нижней гранью $A$ и обозначается
    $\inf A$ (инфимум).
\end{definition}

% \subsection{Метод математической индукции}

% Говорят, что утверждения $A_{n_0}, A_{n_0 + 1}, \dots, A_{n_0 + n}, \dots$ являются
% истинными по методу математической индукции (ММИ), если выполняются два утверждения:
% \begin{enumerate}
%     \item $A_{n_0}$ -- истина (база индукции)
%     \item Если $A_k$ -- истина, то $A_{k+1}$ -- истина (шаг индукции)
% \end{enumerate}

\subsection{Признак верхней и нижней грани множества}

$M = \sup A$, если 
\begin{enumerate}
    \item $\forall x \in A \quad x \le M$
    \item $\forall M' < M \quad \exists x_0 \in A: x_0 > M'$
    (или $\forall \varepsilon > 0 \quad \exists x_\varepsilon \in A: x_\varepsilon > M - \varepsilon$)
\end{enumerate}

$m = \inf A$, если 
\begin{enumerate}
    \item $\forall x \in A \quad x \ge m$
    \item $\forall m' > m \quad \exists x_0 \in A: x < m'$
    (или $\forall \varepsilon > 0 \quad \exists x_\varepsilon \in A: x_\varepsilon < m + \varepsilon$)
\end{enumerate}

\begin{theorem}[О существовании верхней грани (нижней грани)]
    У всякого непустого ограниченного сверху множества существует верхняя грань (нижняя грань).
\end{theorem}

\begin{proof}
    Множество $A$ -- ограничено сверху $\implies \exists b \in \mathbb{R}: \forall x \in A:
    x \le b, A \ne 0 \implies \exists a \in A$ и $a \le b$.

    Рассмотрим такие элементы $x \in A: a \le x \le b$. Рассмотрим их $x_m$ приближения.
    Число таких $x$ может быть бесконечным, но число $x_m$ -- конечно.
    (их не более, чем $(b_m - a_m) * 10^m$). 

    Тогда обозначим $C_m = \max (x_m) (x \in A; a \le x \le b)$.
    Так как $x_{m+1} - x_m < \frac{1}{10^m} \implies C_{m+1} - C_m < \frac{1}{10^m}$
    $\stackrel{\text{по св-ву}}{\implies} C_m$ -- нижние $m$-значные приближения некоторого 
    действительного числа $C$.

    Докажем, что $C = \sup A$ по признаку.
    \begin{enumerate}
        \item Надо доказать $\forall x \in A: x \le C$
        
        Допустим от противного: $\exists x_0 \in A: x_0 > C \stackrel{\text{по опр}}{\implies}
        \exists m_1: \forall m \ge m_1 \quad (x_0)_m > \overline{C_m}$, но
        $\overline{C_m} \ge C_m \implies (x_0)_m > C_m$, а это противоречит выбору $C_m$
        $\implies \forall x \in A: x \le C$.

        \item Возьмём $\forall C' < C \bydef C_m > \overline{C_m'} \ge C'$.
        Но по выбору $C_m = (x_0)_m \quad (x_0 \in A, a \le x_0 \le b)$.
        Имеем: $x_0 \ge (x_0)_m = C_m > \overline{C_m'} \ge C' \implies x_0 > C'$,
        т.е. $\forall C' < C \quad \exists x_0 \in A: x_0 > C'$.
        $\implies$ по признаку верхней грани $C = \sup A$.
    \end{enumerate}
\end{proof}

\underline{\textbf{Замечание 1.}}

В ходе доказательства были получены неравенства: $\forall x \in A$
\[x_m \le (\sup A)_m \le \sup A\]
\[\overline{x_m} \ge \overline{(\inf A)_m} \ge \inf A\]

\parspace

\underline{\textbf{Замечание 2.}}

Если множество $A$ не ограничено сверху, то $\sup A = + \infty$, а если множество $A$
не ограничено снизу, то $\inf A = - \infty$.

\parspace

\begin{theorem}[Свойство граней]
    Если $\forall x \in A \quad \forall y \in B: x \le y \implies \sup A \le \inf B$
\end{theorem}

\begin{proof}
    По условию $\forall y \in B \quad y \ge x \implies x$ -- миноранта $B$
    $\implies \inf B \ge x$, т.е. $\forall x \in A \quad x \le \inf B \implies \inf B$ --
    мажоранта $A$ $\implies \sup A \le \inf B$.
\end{proof}

