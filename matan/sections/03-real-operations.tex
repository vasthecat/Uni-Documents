\section{Арифметические операции для действительных чисел}

\subsection{Свойства арифметических операций}

\begin{definition}
    Рассмотрим множество \underline{всех} таких чисел
    $\alpha_1, \alpha_2, \beta_1, \beta_2$ с конечным числом десятичных знаков
    после запятой, что
    \[\alpha_1 \le x \le \alpha_2\]
    \[\beta_1 \le y \le \beta_2\]

    Тогда суммой чисел $x$ и $y$ ($x, y \in \mathbb{R}$) назовём число 
    $(x + y) = \sup (\alpha_1 + \beta_1) = \inf (\alpha_2 + \beta_2)$

    Пусть теперь $x \ge 0, y \ge 0$, тогда произведением неотрицательное $x$ и $y$
    назовём число $(x * y) = \sup (\alpha_1 * \beta_1) = \inf (\alpha_2 * \beta_2)$.
    Действия с отрицательными числами определим так же, как и для рациональных чисел.

    Разностью $x$ и $y$ ($x, y \in \mathbb{R}$) назовём число $(x - y) = x + (-y)$.

    Частным чисел $x$ и $y$ ($x \in \mathbb{R}, y > 0$) назовём число 
    $\frac{x}{y} = x * \frac{1}{y}$, где $\frac{1}{y} = \inf \frac{1}{\beta_1}$.
    Для $y < 0: \:$ $\frac{x}{y} = -x * \frac{1}{-y}$.
\end{definition}

\underline{\textbf{Свойства арифметических операций}}
\begin{enumerate}
    \item $a + b = b + a$ (коммутативность сложения)
    \item $a * b = b * a$ (коммутативность умножения)
    \item $(a + b) + c = a + (b + c)$ (ассоциативность сложения)
    \item $(a * b) * c = a * (b * c)$ (ассоциативность умножения)
    \item $\forall a \in \mathbb{R} \quad a + 0 = a$ 
    (существование нейтрального элемента сложения)
    \item $\forall a \in \mathbb{R} \quad a * 1 = a$
    (существование нейтрального элемента умножения)
    \item $a + (-a) = 0$
    \item $a * \frac{1}{a} = 1$ ($a \neq 0$)
    \item $(a + b) * c = a * c + b * c$ (дистрибутивность)
    \item Если $a < b \implies a + c < b + c$
    \item Если $a < b, c > 0 \implies a * c < b * c$,
    
    Если $a < b, c < 0 \implies a * c > b * c$
    \item \textbf{Аксиома Архимеда}
    
    \textit{Каково бы ни было действительное число $a$, единицу можно столько
    раз повторить слагаемым, что полученная сумма превзойдет число $a$}
\end{enumerate}

