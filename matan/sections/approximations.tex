\section{Равномерное приближение заданной функции полиномами}

\begin{theorem}[О приближении по $\Delta$-последовательности]
    Пусть функции $D_n(x)$ удовлетворяют условиям:

    \begin{enumerate}
        \item $D_n(x)$ интегрируема на $[-\omega; \omega] \; (\omega > 0)$
        \item $D_n(-x) = D_n(x)$
        \item $D_n(x \geq 0)$
        \item $\ds\int_{-\omega}^\omega D_n(x) dx = 1$
        \item 
            $\{ D_n(x) \}$ равномерно сходится к $0$ на 
            $[-\omega; -\delta] \cup [\delta; \omega] \quad
            \forall \delta \geq 0 \quad \delta < \omega$

            Если $f(x)$ непрерывна на $[a - \omega; b + \omega]$, то
            $\{ f_n(x) \} : f_n(x) = 
            \ds\int_{-\omega}^\omega f(x + t) D_n(t) dt$ сходится равномерно 
            на $[a; b]$ к $f(x)$
    \end{enumerate}
\end{theorem}
\begin{proof}
    
\end{proof}

\begin{remark}
    
\end{remark}


\begin{theorem}[Первая Теорема Вейерштрасса о равномерном приближении 
непрерывной функции алгебраическими полиномами]
    
\end{theorem}
\begin{proof}
    
\end{proof}


\begin{theorem}[Вторая Теорема Вейерштрасса о равномерном приближении
непрерывной периодической функции последовательностью полиномов]
    
\end{theorem}

\begin{remark}
    
\end{remark}


\begin{definition}
    
\end{definition}

\begin{definition}
    
\end{definition}


\begin{theorem}[О связи равномерной сходимости и сходимости в среднем]
    
\end{theorem}
\begin{proof}
    
\end{proof}


\begin{theorem}[О сходимости $\ds\int_a^b f_n(x) g(x) dx$]
    
\end{theorem}
\begin{proof}
    
\end{proof}


\begin{theorem}[О приближении в среднем интегрируемой функции непрерывной]
    
\end{theorem}
\begin{proof}
    
\end{proof}

\begin{corollary}
    
\end{corollary}
\begin{proof}
    
\end{proof}
