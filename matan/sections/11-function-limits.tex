\section{Пределы функций}

\subsection{Пределы функций}

\begin{definition}
    Если каждому $x \in A$ поставить в соответствие единственное число $y \in B$,
    то говорят, что это соответствие задаёт функцию $y = f(x)$.

    Множество $A$ называют \textit{областью определения функции} и обозначают $D_f$.

    Множество $B$ называют \textit{областью значений функции} и обозначают $E_f$.

    $x$ -- аргумент, $y = f(x)$ -- значение функции в точке $x$.
\end{definition}

\parspace

\textbf{Примеры:}
\begin{enumerate}
    \item $A \subset \R, B \subset \R \quad y = f(x)$ -- действительная
    функция действительного аргумента

    \item $A \subset \bb{C}, B \subset \bb{C} \quad w = f(z)$ -- комплексная
    функция комплексного аргумента.

    \item $A \subset \R^m, B \subset \R \quad y = f(x_1, \dots, x_m)$ --
    действительная функция многих переменных.

    \item $A \subset \R^m, B \subset \R^n \quad 
    (y_1, \dots, y_n) = (f_1(x_1, \dots, x_m), f_2(x_1, \dots, x_m), \dots, f_n(x_1, \dots, x_m))$ --
    векторная функция векторного аргумента.
\end{enumerate}

\begin{definition}
    Функция $y = f(x)$ называется ограниченой (сверху, снизу), если её множество
    значений ограничено (сверху, снизу), т.е 
    \[
        \exists m, M: \, \forall x \in D_f \quad m \le f(x) \le M
        \text{ или }
        \exists C: \, \forall x \in D_f \quad  | f(x) | \le C
        \; \text{(ограничено)}
    \]

    \[\exists M: \, \forall x \in D_f \quad f(x) \le M \; \text{(огр. сверху)}\]

    \[\exists m: \, \forall x \in D_f \quad m \le f(x) \; \text{(огр. снизу)}\]
\end{definition}

Пусть функция $y = f(x)$ задана на множестве $A$ и $a$ -- предельная точка $A$.

\begin{definition}[предела функции по Гейне]
    Число $b$ называют пределом функции $f(x)$ в точке $a$ \textbf{в смысле Гейне}
    и обозначают $\dslim_{x \to a} f(x) = b$, если
    $\forall \{ x_n \}: x_n \in A, x_n \ne a, x_n \approach{n \to \infty} a$ выполняется
    $f(x_n) \approach{n \to \infty} b$, то $b$ -- предел функции $f$ в точке $a$.
\end{definition}

\begin{definition}[предела функции по Коши]
    Число $b$ называют пределом функции $f(x)$ в точке $a$ \textbf{в смысле Коши}
    и обозначают $\dslim_{x \to a} f(x) = b$, если
    $\forall \varepsilon > 0 \quad \exists \delta_\varepsilon > 0 \quad
    \forall x \in A : \; 0 < | x - a | < \delta_\varepsilon$ выполняется
    $|f(x) - b| < \varepsilon$

    *рисунок*
\end{definition}

\begin{theorem}[Об эквивалентности определений Коши и Гейне]
    Определение Коши и Гейне предела функции в точке эквивалентны.
\end{theorem}
\begin{proof}
    \begin{enumerate}[label=\alph*)]
        \item 
            Пусть $b = \dslim_{x \to a} f(x)$ в смысле Коши.
            $\bydef$ $\forall \varepsilon > 0 \quad 
            \exists \delta_\varepsilon > 0 \quad \forall x \in A: \; 0 < |x - a| < \delta_\varepsilon$
            выполняется $|f(x) - b| < \varepsilon$.

            Возьмём $\forall \{ x_n \}: x_n \in A, x_n \ne a, \dslimn x_n = a$
            $\bydef$ $\forall \delta > 0 \quad \exists n_\delta \in \N: 
            \forall n \ge n_\delta$ выполняется $| x_n - a | < \delta$

            Возьмём $\delta = \delta_\varepsilon$ и тогда имеем, что $0 < |x_n - a| < \delta_\varepsilon$

            Тогда получим, что $|f(x_n) - b| < \varepsilon \implies 
            \forall \varepsilon > 0 \quad \exists n_\varepsilon = n_\delta \in \N \quad
            \forall n \ge n_\varepsilon$ выполняется $|f(x_n) - b| < \varepsilon
            \bydef \exists \dslim_{n \to b} f(x_n) = b \implies b = \dslim_{x \to a} f(x)$
            в смысле Гейне.

        \item
            Пусть $b = \dslim_{x \to a} f(x)$ в смысле Гейне.
            $\bydef$ $\forall \{ x_n \}: x_n \in A, x_n \ne a, \dslimn x_n = a$
            выполняется $f(x_n) \approach{n \to \infty} b$.

            \textit{Допустим от противного}, $b$ не является пределом $f(x)$ в точке $a$ в смысле Коши:
            $\forall \varepsilon_0 > 0 \quad \forall \delta > 0 \quad \exists x_0 \in A: \; 0 < |x_0 - a| < \delta$,
            но $|f(x_0) - b| \ge \varepsilon_0$

            Возьмём
            \[\delta = 1 \implies \exists x_1 \in A: \; 0 < |x_1 - a| < 1 \text{, но } |f(x_1) - b| \ge \varepsilon_0\]
            \[\delta = \frac{1}{2} \implies \exists x_2 \in A: \; 0 < |x_2 - a| < \frac{1}{2} \text{, но } |f(x_2) - b| \ge \varepsilon_0\]
            \[\dots\]
            \[\delta = \frac{1}{n} \implies \exists x_n \in A: \; 0 < |x_n - a| < \frac{1}{n} \text{, но } |f(x_n) - b| \ge \varepsilon_0\]
            \[\dots\]

            Получим $\{ x_n \}: x \in A, x_n \ne 0$ и $0 < |x_n - a| < \frac{1}{n}$, т.е.
            $x_n \approach{n \to \infty} a$, но
            $f(x_n)$ \cancel{$\approach{n \to \infty}$} $b$ (т.к. $|f(x_n) - b| \ge \varepsilon_0$),
            т.е. определение Гейне не выполняется.

            Получили противоречие $\implies$ $b$ является пределом $f(x)$ в точке $a$ в смысле Коши.
    \end{enumerate}
\end{proof}

\begin{theorem}[Критерий Коши предела функции в точке]
    Для того, чтобы существовал $\dslim_{x \to a} f(x)$ необходимо и достаточно, чтобы
    $\forall \eps > 0 \quad \exists \delta_\eps > 0: \forall x' \in A, x'' \in A:
    \begin{array}{c}
        0 < | x' - a | < \delta_\eps \\
        0 < | x'' - a | < \delta_\eps
    \end{array}$ выполняется $| f(x') - f(x'') | < \eps$.
\end{theorem}
\begin{proof}
    \begin{enumerate}[label=\alph*)]
        \item 
            Необходимость

            Имеем, что $\exists \dslim_{x \to a} f(x) = b$ $\implies$ по определению Коши
            $\forall \eps > 0 \quad \exists \delta_\eps > 0: \forall x \in A: 0 < | x - a | < \delta_\eps$
            выполняется $| f(x) - b | < \frac{\eps}{2}$

            Возьмём $x', x'' \in A:
            \begin{array}{c}
                0 < | x' - a | < \delta_\eps \\
                0 < | x'' - a | < \delta_\eps
            \end{array}$

            Тогда $| f(x') - b | < \frac{\eps}{2}$ и $| f(x'') - b | < \frac{\eps}{2}$

            Рассмотрим $| f(x') - f(x'') | = | (f(x') - b) + (b - f(x'')) | \stackrel{\triangle}{\le}
            | f(x') - b | + | f(x'') - b | < \eps$.

            Получили что и требовалось доказать.
        
        \item
            Достаточность

            Пусть $\forall \eps > 0 \quad \exists \delta_\eps > 0: \forall x', x'' \in A:
            \begin{array}{c}
                0 < | x' - a | < \delta_\eps \\
                0 < | x'' - a | < \delta_\eps
            \end{array}$ выполняется $| f(x') - f(x'') | < \eps$

            Возьмём $\forall \{ x_n \}: \: x_n \in A, x_n \ne a, x_n \approach{n \to \infty} a$
            $\implies$ по критерию Коши $\forall \delta > 0 \quad \exists n_\delta \in \N:
            \forall n \ge n_\delta, \forall m \ge n_\delta$ выполняется $| x_n - x_m | < \delta$.

            Возьмём $\delta = \delta_\eps, x' = x_n, x'' = x_m \implies | x' - x'' | < \delta$.

            Если использовать определение предела, то есть для $\delta > 0 \quad \exists n_{\delta^*} \in \N:
            \begin{array}{c}
                \forall n \ge n_{\delta^*} \text{ выполняется } | x_n - a | < \delta \\
                \forall m \ge n_{\delta^*} \text{ выполняется } | x_m - a | < \delta
            \end{array}$

            $\delta = \delta_\eps, x' = x_n, x'' = x_m$ выполняется
            $\begin{array}{c}
                0 < |x' - a| < \delta_\eps \\
                0 < |x'' - a| < \delta_\eps
            \end{array}$
            $\implies |f(x') - f(x'')| < \eps$ или $|f(x_n) - f(x_m)| < \eps$.
            
            Получили, что $\forall \eps > 0 \quad \exists n_\eps = n_{\delta^*} \in \N: \:
            \forall n \ge n_\eps$ выполняется $|f(x_n) - f(x_m)| < \eps \implies
            \{ f(x_n) \}$ -- фундаментальна $\implies \{ f(x_n) \}$ -- сходится и пусть
            $b = \dslimn f(x_n)$.

            Осталось доказать, что $b$ не зависит от выбора $\{ x_n \}$.

            Пусть $x'_n \approach{} a, x_n'' \approach{} a$. Сопоставим последовательность
            $x'_1, x''_1, x'_2, x''_2, \dots, x'_n, x''_n: \: y_n \implies y_n \approach{} a
            \implies f(y_n)$ -- сходится.

            Но если $\dslimn f(x'_n) = b \implies \dslimn f(x''_n) = b \implies b = \dslim_{x \to a} f(x)$
            в смысле Гейне.
    \end{enumerate}
\end{proof}

\begin{definition}
	Число $b$ называется левым (правым) пределом функции $f(x)$ в точке $x = a$, 
    если $\forall \eps > 0 \quad \exists \delta_\eps > 0: \: \forall x \in A: \: a - \delta_\eps < x < a$ 
    выполняется $|f(x) - b| < \eps \quad (a < x < a + \delta_\eps)$ и обозначается 
    $b = \dslim_{x \to a - 0} f(x) \quad (b = \dslim_{x \to a + 0} f(X))$
\end{definition}

\textbf{Утверждение. }
$\exists \dslim_{x \to a} f(x) - b \iff \exists \dslim_{x \to a - 0} = \dslim_{x \to a + 0} = b$

\begin{definition}
    Пусть $\omega(\omega_1, \omega_2, \dots, \omega_m) \in R^m$ -- произвольный вектор
    единичной длины $|\omega|=1$ и пусть в любой окрестности точки $a \in R^m$ имеются 
    точки вида $x(t) = a + t \omega$ ($t \in R,\, 0 \le t \le t_0$), входящие в область определения $f(x)$.
    Для таких $x(t)$ рассмотрим функцию $ \varphi(t) = f(x(t))$. Если $\exists \dslim_{t \to 0 + 0}  \varphi(t) = b$,
    то его называют пределом $f(x)$ в точке $x = a$ по направлению вектора $\omega$.
\end{definition}


*Рисунок*

\begin{example}
    Если $\exists \dslim_{x \to a} f(x) \quad (x \in R^m) \implies f(x)$ 
    имеет в точке a предел по любому направлению, обратное - неверно!

    Рассмотрим $u(x_1, x_2) = \frac{x_1^2-x_2^2}{x_1^2+x_2^2}; \quad a(0, 0)$

    $\forall\omega \implies$

    \[ x(t) = (t \omega_1; t \omega_2) \]
    \[ 
         \varphi(t) = u(x(t)) = 
        \frac{(t \omega_1)^2 - (t \omega_2)^2}{(t \omega_1)^2 + (t \omega_2)^2} = 
        \frac{t^2 (\omega_1^2 - \omega_2^2)}{t^2 (\omega_1^2 + \omega_2^2)}
    \]
    \[
        \implies \exists \dslim_{t \to +0}  \varphi(t) = 
        \dslim_{t \to +0} \frac{t^2 (\omega_1^2 - \omega_2^2)}{t^2 (\omega_1^2 + \omega_2^2)} = 
        \frac{\omega_1^2 - \omega_2^2}{\omega_1^2 + \omega_2^2}
    \]

    Но для каждого направления этот предел свой $\implies \not\exists \dslim_{x \to a} u(x)$  
\end{example}

\begin{definition}
    Будем обозначать: 
    \begin{enumerate}
        \item $\dslim_{x \to a} f(x) = \infty \iff \forall \eps > 0 \quad \exists \delta_\eps > 0 \quad \forall x \in A:\: 0 < |x-a| < \delta_\eps$ выполняется $|f(x)| > \eps$
        \item $\dslim_{x \to a} f(x) = +\infty \iff \forall \eps > 0 \quad \exists \delta_\eps > 0 \quad \forall x \in A:\: 0 < |x-a| < \delta_\eps$ выполняется $f(x) > \eps$
        \item $\dslim_{x \to a} f(x) = -\infty \iff \forall \eps > 0 \quad \exists \delta_\eps > 0 \quad \forall x \in A:\: 0 < |x-a| < \delta_\eps$ выполняется $f(x) < -\eps$
        \item $\dslim_{x \to \infty} f(x) = b \iff \forall \eps > 0 \quad \exists \delta_\eps > 0 \quad \forall x \in A:\: |x| > \delta_\eps$ выполняется $|f(x)-b| < \eps$
        \item $\dslim_{x \to +\infty} f(x) = b \iff $
        \item $\dslim_{x \to -\infty} f(x) = b \iff $
    \end{enumerate}
\end{definition}

