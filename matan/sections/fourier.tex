\section{Ряд и Интеграл Фурье}

\subsection{Задача, приводящая к ряду Фурье}

Пусть задана $f(x)$ и функциональная последовательность ${\varphi_n(x)}$ на $[a ; b]$.

\textbf{Задача}

Найти такую линейную комбинацию $\varphi_1(x), \varphi_2(x),\dots, \varphi_n(x)$, 
которая лучше всего приближает в среднем функцию $f(x)$ на $[a ; b]$. Т.е. задача 
нахождения таких коэффициентов \[ a_1,\dots,a_n : I = I(a_1,\dots,a_n) = \ds\int_a^b \left[ f(x) 
- \dssum_{k=1}^n a_k\varphi_k(x)\right] ^2 dx \approach{} \text{min.} \]


\begin{definition}
    Функции $\varphi_1(x), \varphi_2(x),\dots$ называются ортогональными на $[a ; b]$,
    если $\ds\int_a^b \varphi_k(x)*\varphi_l(x) dx = 0$
    $\forall k \neq l$.
    
    Тогда мы будем рассматривать ортогональную систему $\varphi_1(x),\dots,\varphi_n(x)$

    \[ I = I(a_1,\dots,a_n) = \ds\int_a^b \left[f(x) - \dssum_{k=1}^n a_k\varphi_k(x)\right]^2 dx 
    = \ds\int_a^b \left(f^2(x) - 2f(x)\dssum_{k=1}^n a_k\varphi_k(x) + \left(\dssum_{k=1}^n a_k\varphi_k(x)\right)^2 \right) dx = \]
    
    \[ = \ds\int_a^b f^2(x) dx - 2\dssum_{k=1}^n a_k \cdot \ds\int_a^b f(x) \varphi_k(x) dx 
    + \ds\int_a^b \left(\dssum_{k=1}^n a_k^2 \varphi_k^2(x) + 2 a_1 a_2 \varphi_1(x) \varphi_2(x)
    +\dots+ 2a_{n-1}a_n\varphi_{n-1}(x)\varphi_n(x) \right) dx = \]
    \[ = \left( \text{в силу ортогональности}: \ds\int_a^b\varphi_k(x)\varphi_l(x) dx = 0 (k \neq l)
    \right) \ds\int_a^b f^2(x)dx - 2 \dssum_{k=1}^n a_k\ds\int_a^b f(x)\varphi_k(x) dx 
    + \dssum_{k=1}^n a_k^2\ds\int_a^b\varphi_k^2(x) dx = \]
    \[ = \ds\int_a^b f^2(x)dx - 2 \cdot \dssum_{k=1}^n a_k\frac{\ds\int_a^b f(x)\varphi_k(x)dx}{\ds\int_a^b \varphi_k^2(x)dx}
    \ds\int_a^b \varphi_k^2(x)dx + \dssum_{k=1}^n a_k^2\ds\int_a^b \varphi_k^2(x) dx = \]
    
    % обозначим
    \[ \left(\dn \frac{\ds\int_a^b f(x)\varphi_k(x)dx}{\ds\int_a^b \varphi_k^2(x)dx} = c_k;
    \ds\int_a^b \varphi_k^2(x) dx = d_k^2 \right) \]
    
    \[ = \ds\int_a^b f^2(x)dx - 2\dssum_{k=1}^n a_k c_k d_k^2 + \dssum_{k=1}^n a_k^2 d_k^2 =
    \ds\int_a^b f^2(x)dx + \dssum_{k=1}^n (a_k - c_k)^2 d_k^2 - \dssum_{k=1}^n c_k^2
    d_k^2 \geq \ds\int_a^b f^2(x)dx - \dssum_{k=1}^n c_k^2 d_k^2 \dn \delta_n \]

    \[ \implies I(a_1,\dots,a_n) \geq \delta_n \text{ и } I(a_1,\dots,a_n) = \delta_n \text{ при } a_k = c_k (k=\overline{1, n}) \]

    т. е. $ \min I(a_1,\dots,a_n) = \delta_n$.

    $\{ \delta_n \}$ --- не возрастает и ограничена снизу $(\delta_n \geq 0)$
    $\iff$ по теореме о сходимости монотонной последовательности
    $\{ \delta_n \}$ сходится, т. е. $\exists \ds\limn \delta_n \dn \delta_0$

    Т. е. $\dslimn \left( \ds\int_a^b f^2(x)dx - \dssum_{k=1}^n c_k^2d_k^2 \right) = \delta_0 $

    \[ \ds\int_a^b f^2(x)dx - \dssum_{k=1}^n c_k^2d_k^2 = \delta_0 \]

    \[ \implies \dssum_{k=1}^n c_k^2d_k^2 = \ds\int_a^b f^2(x)dx - \delta_0 \leq \ds\int_a^bf^2(x)dx. \]

    Это неравенство носит название \textbf{неравенство Бесселя}.

    Оно имеет вид: $\dssum_{k=1}^\infty \frac{\left(\ds\int_a^b f(x)\varphi_k(x)dx\right)^2}{\ds\int_a^b \varphi_k^2(x)dx} \leq \ds\int_a^b f^2(x) dx$.

    Таким образом задача решается для ортогональной системы функций $\{ \varphi_k \}$
    и минимум указанной величине доставляют коэффициенты $a_k$  следующего вида:

    \[ a_k = c_k = \frac{\ds\int_a^b f(x) \varphi_k(x)dx}{\ds\int_a^b \varphi_k^2(x) dx}. \]

    Если $\delta_0 = 0 \implies f_n(x) = \dssum_{k=1}^n c_k\varphi_k(x)$ сходится
    в среднем к $f(x)$ на $[a ; b]$ и в этом случае неравенство Бесселя
    превращается в равенство Парсеваля: 
    \[ \dssum_{k=1}^\infty c_k^2 d_k^2 = \ds\int_a^b f^2(x)dx \]
    
\end{definition}


\begin{definition}
    Если $\delta_0 = 0$ для произвольной функции $f(x)$ из некоторого класса функций,
    то система $\{ \varphi_n(x) \}$ называется полной на отрезке $[a ; b]$ для
    рассматриваемого класса функций.
\end{definition}


\subsection{Тригонометрический ряд Фурье}

Рассмотрим наилучшее приближение на $[a ; b]$ системой тригонометрических полиномов
степени $n$ с периодом $T=b-a$. Т.к. при линейной замене переменных степень полинома
не меняется, то вместо $[a ; b]$ будем рассматривать $[-\pi ; \pi]$ и тригонометрические
полиномы вида:

\[T_n(x) = a_0 + \dssum_{k=1}^n a_k \cos kx + b_k \sin kx \]

\[\ds\int_{-\pi}^\pi (\cos kx \cdot \cos lx) dx =
 \begin{cases}
    0, &k \neq l \\
    \pi, &k = l \neq 0 \\
    2\pi, &k = l = 0   
  \end{cases}
\]

\[ \ds\int_{-\pi}^\pi (\sin kx \cdot \sin lx) dx =
 \begin{cases}
    0, &k \neq l \\
    \pi, &k = l \neq 0
 \end{cases}   
\]

\[\ds\int_{-\pi}^\pi (\sin lx \cdot \cos kx) dx = 0 \]

Таким образом последовательность $1, \cos x, \sin x, \cos 2x, \sin 2x,\dots$ --- 
взаимоортогональные функции на отрезке $[-\pi ; \pi]$, поэтому по выше приведённому
рассуждению, наилучшее приближение в среднем для заданной функции $f(x)$ дают тригонометрические
полиномы вида:

\[f_n(x) = \frac{a_0}{2} + \dssum_{k=1}^n a_k \cos kx + b_k \sin kx \text{, где } \]

\[a_k = \frac{1}{\pi} \ds\int_{-\pi}^\pi f(x) \cos kx \; dx \]

\[b_k = \frac{1}{\pi} \ds\int_{-\pi}^\pi f(x) \sin kx \; dx \]

Тогда эти функции $f_n(x)$ можно рассматривать как частичные суммы ряда:

\[\frac{a_0}{2} + \dssum_{k=1}^\infty a_k \cos kx + b_k \sin kx \] который
называется тригонометрическим рядом Фурье функции $f(x)$. Для такой системы неравенство Бесселя
примет следующий вид:

\[\frac{a_0^2}{2} + \dssum_{k=1}^\infty \left(a_k^2 + b_k^2 \right) \leq \frac{1}{\pi} \ds\int_{-\pi}^\pi f^2(x)dx \]
Покажем, что наша система функций $\frac{1}{2}, \cos x, \sin x, \cos 2x, \sin 2x,\dots$
является полной для интегрируемой на отрезке $[-\pi ; \pi]$ функции $f(x)$

\begin{remark}
    Таким образом, в рассматриваемой задаче ряд 
    $\ds\sum_{k = 1}^\infty c_k \varphi_k(x)$, где
    $c_k = \frac{\ds\int_a^b f(x) \varphi_k(x) dx}{\ds\int_a^b \varphi_k^2(x) dx}$
    является рядом Фурье для функции $f(x)$ по ортогональной системе $\{ \varphi_k(x) \}$.
\end{remark}

\begin{theorem}[Теорема о сходимости в среднем частичных сумм ряда Фурье]
    Частичные суммы ряда Фурье интегригрирумой на $[-\pi ; \pi]$ функции $f(x)$,
    сходятся в среднем на этом отрезке к этой функции.    
\end{theorem}
\begin{proof}
    $f(x)$ интегрируема на $[-\pi ; \pi]$, тогда по теореме о приближении интегрируемой
    функции $\forall \eps_1 > 0 \exists g_{\eps_1}(x)$ --- непрерывная на $[-\pi ; \pi]$

    \[g_{\eps_1}(-\pi) = g_{\eps_1}(\pi) \text{ и } \ds\int_{-\pi}^\pi \left(f(x) - g_{\eps_1}(x) \right)^2dx < \eps_1 \]

    По второй теореме Вейерштрасса существует равномерно сходящаяся
    к $g_{\varepsilon_1}(x)$ последовательность тригонометрических 
    полиномов $\{ T_n(x) \}$ с периодом $2\pi$. Она сходится к 
    $g_{\varepsilon_1}(x)$ и в среднем, то есть $\forall \varepsilon_1 > 0 \quad
    \exists n_1 \: : \quad \forall n \geq n_1 \quad
    \ds\int_{-\pi}^\pi (T_n(x) - g_{\varepsilon_1}(x))^2 dx < \varepsilon_1$.

    Частичная сумма $\{ f_n(x) \}$ даёт для $f(x)$ наилучшее приближение в 
    среднем на $[-\pi; \pi]$ среди всех полиномов $T_n(x) \implies
    \ds\int_{-\pi}^\pi (f(x) - f_n(x))^2 dx \leq 
    \ds\int_{-\pi}^\pi (f(x) - T_n(x))^2 dx \implies$ при $n \geq n_1$:
    
    \begin{align*}
        \ds\int_{-\pi}^\pi (f(x) - f_n(x))^2 dx &\leq 
        \ds\int_{-\pi}^\pi (f(x) - T_n(x))^2 dx
        = \ds\int_{-\pi}^\pi [ (f(x) - g_{\varepsilon_1}(x)) + (g_{\varepsilon_1}(x) - T_n(x)) ]^2 dx \leq \\
        &\leq 2 \ds\int_{-\pi}^\pi (f(x) - g_{\varepsilon_1}(x))^2 dx +
        2 \ds\int_{-\pi}^\pi (g_{\varepsilon_1}(x) - T_n(x))^2 dx < 4 \varepsilon_1 \dn \varepsilon
    \end{align*}

    Получаем, что $\forall \varepsilon > 0 \quad \exists n_\varepsilon = n_1 :
    \forall n \geq n_\varepsilon \quad \ds\int_{-\pi}^\pi (f(x) - f_n(x))^2 dx < \varepsilon$
    то есть частичная сумма ряда Фурье $\{ f_n(x) \}$ сходится в среднем к $f(x)$.
\end{proof}

\begin{remark}
    \begin{enumerate}
        \item
            Система $1, \cos x, \sin x, \cos 2x, \dots$ является полной на
            классе интегрируемых на $[-\pi; \pi]$ функций.
        \item
            Так как $f_n(x)$ имеют на $[-\pi; \pi]$ непрерывную производную
            любого порядка, получаем, что интегрируемую на $[-\pi; \pi]$
            функцию можно приблизить в среднем на нём с помощью функции,
            имеющей непрервыную производную любого порядка.
    \end{enumerate}
\end{remark}