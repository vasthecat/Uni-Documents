\section{Функциональные ряды}

\begin{definition}
    Пусть функции $\varphi_k(x)$ заданы на $A$. Тогда выражение
    \[
        \varphi_1(x) + \varphi_2(x) + \dots + \varphi_n(x) + \dots = 
        \series \varphi_k(x)
    \]
    называют функциональным рядом.
\end{definition}

\begin{definition}
    Если числовой ряд $\series \varphi_k(x_0), \, x_0 \in A$ сходится, то
    говорят, что функциональный ряд сходится в точке $x_0$. Множество таких
    $x_0$ составляет область сходимости функционального ряда.
\end{definition}

\begin{definition}
    Говорят, что функциональный ряд $\series \varphi_k(x)$ сходится равномерно
    на множестве $A$, если функциональная последовательность частичных сумм
    этого ряда $S_n(x) = \dssum_{k=1}^n \varphi_k(x)$ равномерно сходится
    на множестве $A$.
\end{definition}

\begin{remark}
    Благодаря связи сходимости функционального ряда со сходимостью 
    функциональной последовательности на функциональные ряды переносятся все
    теоремы о функциональных последовательностях.
\end{remark}

\begin{theorem}[Критерий Коши равномерной сходимости функционального ряда]
    Справедливы следующие утверждения:

    \begin{enumerate}
        \item
            Для того, чтобы $\series \varphi_k(x)$ равномерно сходился на $A$ $\iff$
            $\forall \varepsilon > 0 \quad \exists n_\varepsilon: \: 
            \forall n \geq n_\varepsilon \quad \forall m \geq n_\varepsilon \quad
            \forall x \in A$ выполняется 
            $\left| \dssum_{k=n+1}^m \varphi)k(x) \right| < \varepsilon$
        
        \item
            Если $\series \varphi_k(x)$ равномерно сходится на $A$ к сумме
            $S(x)$ и $\exists \dslim_{x \to a} \varphi_k(x) = b_k$, то ряд
            $\series b_k$ сходится и $\exists \dslim_{x \to a} S(x) =
            \series b_k = \series \left( \dslim_{x \to a} \varphi_k(x) \right)$

        \item 
            Если $\series \varphi_k(x)$ равномерно сходится на $A$ к сумме
            $S(x)$ и каждая функция $\varphi_k(x) \quad \forall k \in \N$ --
            непрерывна в точке $a \in A \implies S(x)$ непрерывна в точке $a \in A$.
        
        \item
            Если функциональный ряд $\series \varphi_k(x)$ равномерно сходится
            на $[a; b]$ к $S(x)$ и $\varphi_k(x)$ -- интегрируема на $[a; b]$
            $\forall k \in \N \implies S(x)$ интегрируема на $[a; b]$ и
            $\ds\int_a^b S(x) dx = \series \int_a^b \varphi_k(x) dx$

            $\left( \ds\int_a^b \left(\series \varphi_k(x)\right) dx = \series \int_a^b \varphi_k(x) dx \right)$

        \item
            Если функции $\varphi_k(x)$ дифференцируему на $[a; b]$
            $\forall k \in \N$ и ряд $\series \varphi'_k(x)$ равномерно
            сходится на $[a; b]$, то из сходимости $\series \varphi_k(x)$ в
            некоторой $x_0 \in [a; b] \implies \series \varphi_k(x)$ сходится
            на $[a; b]$ к некоторой функции $S(x)$, причём $S(x)$ 
            дифференцируема на $[a; b]$ и $S'(x) = \series \varphi_k(x)$.

            $\left( \left( \series \varphi_k(x) \right)' = \series \varphi'_k(x) \right)$
    \end{enumerate}
\end{theorem}

\begin{theorem}[Признак Вейерштрасса равномерной сходимости функционального ряда]
    Если все функции 
    $\forall k \in \N \quad \forall x \in A \quad |\varphi_k(x)| \leq a_k$ 
    и числовой ряд $\series a_k$ сходится $\implies \series \varphi_k(x)$
    сходится равномерно на $A$.
\end{theorem}
\begin{proof}
    $\series a_k$ сходится $\implies$ по критерию Коши, $\forall \varepsilon > 0
    \quad \exists n_\varepsilon \quad \forall n \geq n_\varepsilon \quad
    \forall p \in \N$ выполняется $\left| \dssum_{k=n+1}^{n+p} a_k \right| < \varepsilon$

    Тогда $\forall x \in A \quad \left| \dssum_{k=n+1}^{n+p} \varphi_k(x) \right|
    \leq \dssum_{k=n+1}^{n+p} |\varphi_k(x)|
    \leq \dssum_{k=n+1}^{n+p} a_k < \varepsilon$
    $\implies$ по критерию Коши равномерной сходимости функционального ряда,
    $\series \varphi_k(x)$ равномерно сходится на $A$.
\end{proof}

\begin{definition}
    Ряд $\series a_k \cdot (x - x_0)^k$ называется степенным рядом
\end{definition}

\begin{definition}
    Радиусом сходимости степенного ряда $\series a_k (x - x_0)^k$ называют
    $R = \sup \{ r: \: r > 0 \; \& \; a_n r^n = O(1) \}$, то есть огр. ( ??? )

    В этом определении может быть так, что $R = +\infty$.
\end{definition}

\begin{theorem}[о множестве сходимости степенного ряда]
    Если $R > 0$ и $0 < r < R$ $\implies \series a_k (x - x_0)^k$ сходится
    абсолютно на $K_r = \{ (x - x_0) < r \}$
\end{theorem}
\begin{proof}
    Возьмём $r_1 > r: \: a_n r_1^n = O(1)$ и $\forall x \in K_r$ рассмотрим
    \[ 
        |a_k (x - x_0)^k| \leq |a_k| \cdot r^k
        = |a_k| \cdot r_1^k \, \frac{r^k}{r_1^k}
        \leq C \left( \frac{r}{r_1} \right)^k = C q^k
    \]

     и $\series C q^k$ сходится (бесконечно убывающая геометрическая прогрессия)
    $\implies$ по признаку Вейерштрасса, $\series a_k (x - x_0)^k$ сходится
    равномерно и абсолютно на $K_r$.
\end{proof}

\begin{definition}
    Множество $K_R$ называют кругом сходимости степенного ряда.
\end{definition}

\begin{corollary}
    Если $R > 0$, то $\series a_k (x - x_0)^k$ сходится поточечно в круге $K_R$.
    И при $R < +\infty$ вне круга $K_R$ ряд расходится.
\end{corollary}

\begin{corollary}
    Если $R > 0$ и $|x - x_0| < R \implies a_k (x - x_0)^k = o(1)$, а
    если $R < +\infty$ и $|x - x_0| > R \implies a_k (x - x_0)^k \neq o(1)$
    ($\approach{} 0$)
\end{corollary}

\begin{corollary}
    Если $R > 0$ и $0 < r < R$, то $\implies \series a_k (x - x_0)^k$
    равномерно сходится в $K_r$.
\end{corollary}

\begin{theorem}[Абеля]
    Если числовой ряд $\series a_k (x_1 - x_0)^k$ сходится при $x_1 \neq x_0$,
    то $\series a_k (x - x_0)^k$ сходится в круге 
    $K = \{ |x - x_0| < |x_1 - x_0| \}$.
\end{theorem}
\begin{proof}
    Если ряд $\series a_k (x_1 - x_0)^k$ сходится $\implies x_1 \in K_R$ --
    кругу сходимости 
    $\series a_k (x - x_0)^k \implies |x_1 - x_0| < R \implies k \subset K_R$
\end{proof}

\begin{theorem}[Коши-Адамара]
    Пусть $R$ -- радиус сходимости $\series a_k (x - x_0)^k$ и 
    $l = \ds\overline{\lim_{k \to \infty}} \sqrt[k]{|a_k|}$.
    Тогда $R = \frac{1}{l}$
\end{theorem}
\begin{proof}
    $\ds\overline{\lim_{k \to \infty}} \sqrt[k]{|a_k (x -x_0)^k|}
    = l \cdot |x - x_0|$. Зафиксируем точку $x$ и проведём исследование
    сходимости числового ряда по признаку Коши. Если $l = 0 \implies
    \series a_k (x - x_0)^k$ сходится $\forall x \implies R = \infty$.

    Если $l = +\infty \implies a_k (x - x_0)^k \neq O(1)$ (не ограничено)
    $\implies$ по следствию 2 Теоремы о множестве сходимости, $R = 0$.

    Если $x \in K_{\frac{1}{l}}$, то имеем $l \cdot |x - x_0| < 1 \implies
    \ds\overline{\lim_{k \to \infty}} \sqrt[k]{|a_k (x -x_0)^k|} < 1$
    $\implies$ по признаку Коши, ряд сходится в точке $x$
    ($\forall x \in K_{\frac{1}{l}}$) 
    $\implies K_{\frac{1}{l}} \subset K_R \implies \frac{1}{l} \leq R$.

    Но если $|x - x_0| > \frac{1}{l} \implies a_k (x - x_0)^k \neq o(1)$
    (не $\approach{} 0$) $\implies$ по следствию 3, $\frac{1}{l} \geq R$.

    Следовательно, $\frac{1}{l} = R$.
\end{proof}

\begin{remark}
    Если применить Теорему о почленном дифференцировании и интегрировании к
    степенному ряду, то получим, что эти операции можно проводить для
    степенного ряда на его круге сходимости и при этом ряды, полученные в
    результате применения этих операций также являются степенными и имеют
    тот же круг сходимости.
\end{remark}